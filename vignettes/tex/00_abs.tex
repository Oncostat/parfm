Frailty models are getting more and more popular to account for overdispersion and/or clustering in survival data.
When the form of the baseline hazard is somehow known in advance, the parametric estimation approach can be used advantageously.
Nonetheless, there is no unified widely available software that deals with the parametric frailty model.
The new \pkg{parfm} package remedies that lack by providing
    a wide range of parametric frailty models in \proglang{R}.
The available baseline hazard failies are:
    exponential, Weibull, inverse Weibull (Fr\'echet), Gompertz,
    lognormal, log-kewNormal, and loglogistic.
The gamma, positive stable, inverse Gaussian, and lognormal
    frailty distributions can be specified, together with five different baseline hazards.
Parameter estimation is done by maximising the marginal log-likelihood, with right-censored and possibly left-truncated data.
In the multivariate setting, the inverse Gaussian may encounter numerical difficulties with a huge number of events in at least one cluster.
The positive stable model shows analogous difficulties but an ad-hoc solution is implemented, whereas the gamma model is very resistant due to the simplicity of its Laplace transform.