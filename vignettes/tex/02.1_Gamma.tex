A gamma frailty term is a random variable $U \sim \mathrm{Gam}^\star (\theta)$ with probability density function
\begin{equation*}
% \label{eq:model.gamma}
f ( u ) = \cfrac{
    \theta^{-\tfrac{1}{\theta}} u^{\tfrac{1}{\theta} - 1} \exp \left( - u / \theta  \right)
  }{
    \Gamma ( 1 / \theta )
  }\ccom \qquad \theta > 0,
\end{equation*}
  where $\Gamma ( \cdot )$ is the gamma function.
  It corresponds to a gamma distribution $\mathrm{Gam}(\mu,\theta)$ with $\mu$ fixed to 1
  for identifiability.
Its variance is then $\theta$. 

The associated Laplace transform is given by
\begin{equation*}
%\label{eq:LT.gamma}
  \mathcal L(s) = (1 + \theta s )^{- \tfrac{1}{\theta}},
  \qquad s \ge 0,
\end{equation*}
and it is easy to show that, for $q \geq 1$,
\[
  \mathcal{L}^{(q)} ( s ) = ( - 1 )^{q} \left( 1 + \theta s \right)^{- q} 
    \left[ \prod_{l=0}^{q - 1} (1 + l \theta) \right]
    \mathcal{L} ( s ).
\label{gamma}       
\]
Therefore, in Equation~\ref{eq:loglik.marg}, we have
\begin{equation}
\label{eqn:gamma}
\log \left( ( - 1 )^{q} \mathcal{L}^{( q )} ( s ) \right) = - \left( q + \frac{1}{\theta} \right) \log ( 1 + \theta s ) + \sum_{l = 0}^{q - 1} \log ( 1 + l \theta ).
\end{equation}

%For the gamma distribution, the Kendall's tau, 
%  which translates the frailty distribution parameter 
%  in terms of association between clustered times, 
%  is 
%\begin{equation*}
%  \tau = \frac\theta{\theta + 2} \in (0, 1).
%\end{equation*}

For the gamma distribution, the Kendall's tau \cite[Section~4.2]{Hougaard00},
	which measures the association between any two event times from the same cluster in the multivariate case, 
	can be computed as
\begin{equation*}
\tau = \frac\theta{\theta + 2} \in (0, 1).
\end{equation*}