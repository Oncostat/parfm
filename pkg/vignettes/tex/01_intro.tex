% % % General intro
% % Survival
Survival data, or time-to-event data, measure the time 
  elapsed from a given origin to the occurrence of an event of interest.
The observation of survival data is very common in the medical fields
  where, for instance, the clinician is interested in the time to relapse of a pathology after the therapy.
However, the researcher cannot always observe the event due to censoring. 
Right-censoring occurs when the time of interest cannot be observed but only a lower bound is available. 
Particular techniques are therefore required as described by a number of textbooks, e.g., \cite{KleinMoeschberger2003}.

Most commonly, survival data 
are handled by means of the proportional hazards regression model popularised by \cite{Cox72}.
But correct inference based on those proportional hazards models needs independent and identically distributed samples.
Nonetheless, subjects may be exposed to different risk levels, even after controlling for known risk factors;
	this is because some relevant covariates are often unavailable to the researcher or even unknown (univariate case).
Also, the study population may be divided into clusters so that subjects from the same cluster behave more cohesively than subjects from different clusters (multivariate case).
Lots of examples of clustered survival data arise from large-scale clinical trials
  in which patients are recruited at several hospital centres \citep{DuchateauEtal02, GliddenVittinghoff04}.
  Another classical example
  is the analysis of lifetimes of matched human organs such as eyes or kidneys.


% % Frailty models 
The frailty model, introduced in the biostatistical literature by 
  \cite{VaupelEtal79}, and discussed in details
  by \cite{Hougaard00}, \cite{DuchateauJanssen08}, and by \cite{Wienke10}, 
  accounts for this heterogeneity in baseline.
It is an extension of the proportional hazards model in which the hazard function
    depends upon an unobservable random quantity, the so-called frailty, that acts multiplicatively on it.
  
The gamma frailty model assumes a gamma distribution for the frailties.
Arguably, this is the most popular frailty model due to its mathematical tractability.
The lognormal frailty model is also well-liked for its strong link with generalised linear mixed models.
Other frailty distributions include the positive stable and the inverse Gaussian.
All of these are reviewed by \citeauthor{DuchateauJanssen08} (\citeyear{DuchateauJanssen08}, Chapter~4).

Of particular interest in the multivariate case is the association between related event times.
Indeed, different dependence structures result from different frailty distributions \citep{Hougaard95}.
In particular, positive stable frailties typically generate very strong dependence initially while, at equal global dependence,
gamma frailties lead to stronger dependence at late times, and inverse Gaussian frailties 
  are in between the two.  
% 	induce stronger dependence in the central area.
These three distributions therefore cover a wide range of association structures in the data.

% % Parametric vs. Semiparametric
Estimation of the frailty model can be parametric or semi-parametric. 
In the former case, a parametric density is assumed for the event times, 
  resulting in a parametric baseline hazard function.
Estimation is then conducted by maximising the marginal log-likelihood (see Section~\ref{sec:model}).
In the second case, the baseline hazard is left unspecified and more complex techniques are available to
  approach that situation \citep{AbrahantesEtal07}.
Even though semi-parametric estimation offers more flexibility,
  the parametric estimation will be more powerful if the form of the baseline hazard is somehow known in advance.
Further, the estimation technique is much simpler.


% % Existing software on parametric frailty models
Slowly but surely, a variety of estimation procedures becomes available in standard statistical software.
In \proglang{R} \citep{R}, the \code{coxph()} function from the \pkg{survival}~package \citep{R:survival}
  handles the semi-parametric model with gamma and lognormal frailties.
Important options supported by \code{coxph()} and its output are described in details by 
  \citeauthor{TherneauGrambsch00} (\citeyear{TherneauGrambsch00}, Chapter~9).
Recently, the \pkg{frailtypack}~package \citep{R:frailtypack} by \cite{RondeauGonzales05}
  and \cite{RondeauEtal12} has been updated and it stands now for gamma frailty models
  with a semi-parametric estimation but also with a parametric approach using the Weibull baseline hazard.
Other \proglang{R} packages include \pkg{coxme} \citep{R:coxme} and \pkg{phmm} \citep{R:phmm}. 
These two perform semi-parametric estimation in the lognormal frailty model.
\proglang{SAS} \citep{SAS} also deals with the lognormal distribution. 
On the one hand, \code{proc phreg} can now fit the semi-parametric lognormal frailty model.
On the other hand, \code{proc nlmixed} deals with the parametric version by using Gaussian quadrature to approach the marginal likelihood;
  see, e.g., \citeauthor{DuchateauJanssen08} (\citeyear{DuchateauJanssen08}, Example~4.16).
In the parametric setting, \proglang{STATA} \citep{STATA} provides some flexibility.
The \code{streg}~command \citep{Gutierrez02} is able to perform maximum likelihood estimation 
  with various choices of baselines:
  exponential, Weibull, Gompertz, lognormal, loglogistic, and generalised gamma.
Take notice, however, that \proglang{STATA} fits the accelerated failure time model.
Still, with exponential or Weibull baselines, both the proportional hazards and the accelerated failure time
  representations are allowed.   
As for the frailty distribution, the gamma and the inverse Gaussian are the only two that are supported.
% See \cite{Gutierrez02} for an illustration of \code{streg}.
On a side note, Bayesian analyses can be conducted in \proglang{WinBUGS} \citep{Winbugs};
  see, e.g., \citeauthor{DuchateauJanssen08} (\citeyear{DuchateauJanssen08}, Example~6.4).
For a deeper overview of who supports what, and for a comparison of some of the aforementioned functions,
  see \cite{WienkeHirsch11}.


% % Objective
Hereinbelow, we illustrate \pkg{parfm} \citep{R:parfm}, a new \proglang{R} package that fits 
    the gamma, the positive stable, the inverse Gaussian, and lognormal
    proportional hazards frailty models 
    with either exponential, Weibull, Gompertz, lognormal, 
    or loglogistic baseline.
The main advantage of \pkg{parfm} therefore relies on the large choice 
  of frailty distributions and parametric baseline hazards it supports.
Parameter estimation is done by maximising the marginal log-likelihood.
% % Outline

The model and the marginal log-likelihood are shown in Section~\ref{sec:model}.
There, we also outline the estimation method,
  while Sections~\ref{sec:model:gamma}--\ref{sec:model:LN} provide details 
  for the three frailty distributions supported by \pkg{parfm}.
In Section~\ref{sec:rexample}, we apply \pkg{parfm} to a real dataset in order to illustrate its use and its output.
Section~\ref{sec:concl} concludes with remarks.
