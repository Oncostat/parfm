From a modelling point of view, the multivariate model includes the univariate.
Because of this, we shall mainly refer to the first.
However, they are used in two different contexts: 
	in the former case, the frailty distribution variability is related to a measure of dependence between clustered subjects,
	whereas it is rather interpreted as a measure of overdispersion in the latter.

\paragraph{Model.}
The frailty model is defined in terms of the conditional hazard 
\begin{equation*}
  h_{ij}(t \mid u_i) = h_0(t) u_i \exp{(\bm{x}_{ij}^\top \bm\beta)},
  %\label{eq:01:fm}
\end{equation*}
with  $i \in I = \{ 1, \ldots, G \}$ and $j \in J_{i} = \{ 1, \ldots, n_i \}$, where
      $h_0( \cdot )$ is the baseline hazard function,
      $u_i$ the frailty term of all subjects in group $i$,
      $\bm{x}_{ij}$ the vector of covariates for subject $j$ in group $i$, and
      $\bm\beta$ the vector of regression coefficients.

If the number of subjects $n_i$ is 1 for all groups, then the univariate frailty model is obtained
  \cite[Chapter~3]{Wienke10},
otherwise the model is called the {shared} frailty model (\citeauthor{Hougaard00} \citeyear{Hougaard00}, Chapter~7; \citeauthor{DuchateauJanssen08} \citeyear{DuchateauJanssen08})
  because all subjects in the same cluster share the same frailty value $u_i$.


\paragraph{Baseline hazard.}
Under the parametric approach, the baseline hazard is defined as a parametric
  function and the vector of its parameters, say $\bm{\psi}$,
  is estimated together with the regression coefficients and the frailty parameter(s).
A bunch of possibilities are considered in the literature;
in the \pkg{parfm} package the
  exponential,
  Weibull,
  inverse Weibull (Fr\'echet),
  Gompertz,
  lognormal, 
  logSkewNormal \citep{Azzalini85}, and
  loglogistic
  distributions are available.
Table \ref{tab:model.baselines} shows the hazard and cumulative hazard functions for each of these distributions.

  % \documentclass[11pt, a4paper]{article}
% 
% \usepackage{a4wide}
% 
% \usepackage[english]{babel}
% \usepackage[utf8x]{inputenc}
% \usepackage{ucs}         %This package contains support for using UTF-8 as input encoding in LaTeX documents.
% \usepackage[T1]{fontenc}
% \usepackage{lmodern} 
% 
% \usepackage{amsmath,     %Defines extra environments for multiline displayed equations, as well as a number of other enhancements for math 
%             amssymb,     %Loads lots of extra symbols.
%             amsthm,      %Provides a proof environment and extensions for the \newtheorem command. 
%             bm,          %Allows you to write bold greek letters
%             graphicx, xcolor, multirow, longtable, placeins, url, natbib, ulem, listings, multicol}
% 
% % ------ headers settings ------
% \RequirePackage{fancyhdr}
% \pagestyle{fancy}
% 
% \addtolength{\headheight}{\baselineskip}
% \renewcommand{\sectionmark}[1]{\markboth{#1}{}}
% \renewcommand{\subsectionmark}[1]{\markright{#1}}
% \renewcommand{\headrulewidth}{1.14pt}
% 
% \lfoot{}
% \cfoot{}
% \rfoot{}
% \lhead{}
% \chead{}
% \rhead{$\blacklozenge$\space\textbf{\thepage}}
% % ------------------------------
% 
% \begin{document}

\begin{table}[] \centering 
\renewcommand{\arraystretch}{1.5}
  \begin{tabular}{cccc}
  \hline \hline
  {\bf distribution}                 & $\bm{h_{0}(t)}$                                            & $\bm{H_{0}(t)} = \int_{0}^{t} h_{0} ( s ) \textrm{d} s$           & {\bf parameters space} \\  
  \hline
  {\bf exponential}                  & $\lambda$                                                  & $\lambda t$                                                       & $\lambda > 0$ \\
  {\bf Weibull}                      & $\lambda \rho t^{\rho - 1}$                                & $\lambda t^{\rho}$                                                & $\rho, \lambda > 0$   \\
  {\bf Gompertz}                     & $\lambda \exp ( \gamma t )$                                & $\frac{\lambda}{\gamma} \left(\exp (\gamma t) - 1\right)$         & $\gamma, \lambda > 0$ \\
  \multirow{2}{*}{{\bf lognormal}}   & \multirow{2}{*}{$\frac{\phi \left(\tfrac{\log(t) - \mu}{\sigma} \right)}
	      		   {\sigma t \left[1 - \Phi\left(\tfrac{\log(t) - \mu}{\sigma} \right) \right]}$} & \multirow{2}{*}{$- \log \left[1 - \Phi\left(\tfrac{\log(t) - \mu}
					                                                                                   {\sigma} \right)\right]$}                                      & \multirow{2}{*}{$\mu \in                       \mathbb{R}$, $\sigma > 0$}\\
					                 &                                                            &                                                                   &                         \\     
  \multirow{2}{*}{{\bf loglogistic}}& \multirow{2}{*}{$\cfrac{\exp(\alpha) \kappa t^{\kappa - 1}}
	                                               {1 + \exp(\alpha) t^{\kappa}}$}                & \multirow{2}{*}{$\log \left(1 + \exp(\alpha) t^{\kappa}\right)$} & \multirow{2}{*}{$\alpha \in \mathbb{R}$, $\kappa > 0$} \\
	                                 &                                                            &                                                                   &                          \\                
	\hline \hline
	\end{tabular}
\caption{Parametric distributions available in \pkg{parfm} for the baseline hazard. With the lognormal, $\phi(\cdot)$ and $\Phi(\cdot)$ respectively denote the probability density and the cumulative distribution functions of a standard normal random variable.}
\label{tab:model.baselines}	
\end{table} 

% \end{document}


\paragraph{Frailty distribution.}
The frailty $u_i$ is an unobservable realisation of a random variable $U$
  with probability density function $f(\cdot)$---the frailty distribution.
Since $u_i$ multiplies the hazard function, $U$ has to be non-negative.
Another constraint is further needed for identifiability reasons,
  similar to the zero-mean constraint of a random effect in a standard linear mixed model.
More specifically, the mean of $U$ is typically restricted to unity when possible (i.e., when $\E(U)$ exists)
in order to separate the baseline hazard from the overall level of the random frailties.


Various frailty distributions have been proposed in the literature \cite[Chapter~4]{DuchateauJanssen08}.
Hereinafter, we shall focus on
  the gamma, the positive stable, and the inverse Gaussian frailty distributions.
In all of these three, a single heterogeneity parameter (denoted either $\theta$ or $\nu$)
  indexes the degree of dependence.
In the following, $\xi$ is used as a generic notation to denote either $\theta$ or $\nu$.



\paragraph{Data.}
For right-censored clustered survival data,
  the observation for subject $j \in J_i = \{1, \ldots , n_i \}$
  from cluster $i \in I = \{1, \ldots , G\}$ is the couple $\bm z_{ij} = (y_{ij} , \delta_{ij} )$,
  where $y_{ij} = \min(t_{ij}, c_{ij})$ is the minimum between the survival time $t_{ij}$
    and the censoring time $c_{ij}$,
  and where $\delta_{ij} = I(t_{ij} \le c_{ij})$ is the event indicator.
  Covariate information may also have been collected; in this case,
  $\bm z_{ij} = (y_{ij} , \delta_{ij}, \bm{x}_{ij} )$,
  where $\bm x_{ij}$ denote the vector of covariates for the $ij$-th observation.
Further, if left-truncation is also present,
  truncation times $\tau_{ij}$ are gathered in the vector $\bm{\tau}$.



\paragraph{Likelihood.}
In the parametric setting, estimation is based on the marginal likelihood
  in which the frailties have been integrated out by averaging the conditional likelihood
  with respect to the frailty distribution.
Under assumptions
  of non-informative right-censoring and of
  independence between the censoring time and the survival time random variables,
  given the covariate information,
  the marginal log-likelihood of the observed data $\bm z = \{ \bm z_{ij} ; i\in I, j \in J_i \}$
  can be written as \citep{vandenBergDrepper12}
%   (see page~120 of \citeauthor{DuchateauJanssen08} \citeyear{DuchateauJanssen08} and
%     Section~2.2 of \citeauthor{RondeauEtal03} \citeyear{RondeauEtal03})
%     Section~3.1 of \citeauthor{Gutierrez02} \citeyear{Gutierrez02})
\begin{align}
  \ell_{\mathrm{marg}}(\bm\psi, \bm\beta, \xi; \bm z \mid \bm \tau) =
    \sum_{i=1}^G &\left\{
      \left[ \sum_{j=1}^{n_i}
        \delta_{ij} \left( \log(h_0(y_{ij})) + \bm x_{ij}^\top\bm\beta \right)
      \right]\right.
      \nonumber \\
      &\quad + \log \left[ (-1)^{d_i} \mathcal L^{(d_i)} \left(
        \sum_{j=1}^{n_i} H_0(y_{ij}) \exp(\bm x_{ij}^\top\bm\beta)
      \right) \right]
      \nonumber \\
      &\quad \left.-\log \left[ \mathcal L\left(
        \sum_{j=1}^{n_i} H_0(\tau_{ij}) \exp(\bm x_{ij}^\top\bm\beta)
      \right) \right]
    \right\},
  \label{eq:loglik.marg}
\end{align}
  with $d_i = \sum_{j=1}^{n_i} \delta_{ij}$ the number of events in the $i$-th cluster, and
  $\mathcal L^{(q)}(\cdot)$ the $q$-th derivative of the Laplace transform of the frailty distribution
    defined as
  \[
    \mathcal{L} ( s ) = \E\big[ \exp(-Us) \big] =
      \int_{0}^{\infty} \exp ( - u_{i} s ) f ( u_{i} ) \mathrm{d}u_{i}, \qquad s \geq 0.
  \]

\paragraph{Estimation.}
Estimates of $\bm{\psi}$, $\bm{\beta}$, and $\bm{\xi}$ are obtained by maximising
  the marginal log-likelihood~\ref{eq:loglik.marg};
this can easily be done if one is able to compute higher order derivatives
  $\mathcal{L}^{( q )} ( \cdot )$ of the Laplace transform up to $q = \max \{ d_{1}, \ldots, d_{G} \}$.
Symbolic differentiation might be performed in \proglang{R},
  but is impractical here, mainly because this is very time consuming.
Therefore, explicit formulas are rather desirable.
Further, they will be used in the calculation of predictions as shown below.

\paragraph{Prediction.}
Besides parameter estimates, prediction of frailties are sometimes desirable.
  As an aside, they are needed at each expectation step of the expectation-maximisation (EM) algorithm that fits
  the semi-parametric frailty model.

The frailty term $u_i$ can be predicted by
  $\hat u_i = \E\left( U \mid \bm z_i, \bm \tau_i ; \hat{\bm\psi}, \hat{\bm\beta}, \hat\xi \right)$,
  with $\bm z_i$ and $\bm \tau_i$ the data and the truncation times of the $i$-th cluster.
This conditional expectation can be achieved as
  \begin{equation*}
    \E\left( U \mid \bm z_i, \bm \tau_i ; \bm\psi, \bm\beta, \xi \right) =
    - \frac{
      \mathcal L^{(d_i + 1)} \left(
          \sum_{j=1}^{n_i}  H_0(y_{ij})
            \exp(\bm x_{ij}^\top \bm\beta)
        \right)
    }{
      \mathcal L^{(d_i)} \left(
          \sum_{j=1}^{n_i} H_0(y_{ij})
            \exp(\bm x_{ij}^\top \bm\beta)
        \right)
    }\ccom
  \end{equation*}
  which can be seen from Appendix~\ref{app:condEfrailty}, together with
  $\E\big[ U^q \exp(-Us) \big] = (-1)^q \mathcal L^{(q)}(s)$.


\paragraph{Outline.}
In Sections~\ref{sec:model:gamma}--\ref{sec:model:LN} we illustrate the three frailty distributions
  which are available in the \pkg{parfm} package:
  the gamma, the positive stable, the inverse Gaussian, and the lognormal.
Note that the Laplace transform of a lognormal random variable does not exist in a closed form.
Hence, Equation~\ref{eq:loglik.marg} requires numerical approximation in that case, which is not considered here.
