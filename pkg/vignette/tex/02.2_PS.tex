\citeauthor{Hougaard00} (\citeyear{Hougaard00}, Section A.3.3)
  introduces the positive stable distributions as a family with two parameters:
  a scale $\delta>0$ and the so-called index $\alpha<1$.
Imposing $\delta=\alpha$, 
  the positive stable frailty distribution $\mathrm{PS}^\star(\nu)$ is obtained,
  with $\nu = 1-\alpha$.

The associated probability density function is then
\begin{equation*} 
%\label{eq:model.PS}
  f(u) = - \frac1{\pi u} \sum_{k=1}^{\infty} \frac{\Gamma ( k (1 - \nu ) + 1 )}{k!} 
    \left( - u^{ \nu - 1} \right)^{k} \sin ( ( 1 - \nu ) k \pi ),
    \qquad \nu \in (0,1).
\end{equation*}
The mean and variance are both undefined. 
Therefore, the heterogeneity parameter $\nu$ does not correspond to the variance of the frailty term.
Because of that, we intentionally call it $\nu$ instead of $\theta$ to avoid misinterpretation.

In contrast to the probability density function,
  the associated Laplace transform takes a very simple form,
\begin{equation*} 
%\label{eq:LT.PS}
  \mathcal L (s) = \exp \left( - s^{1 - \nu} \right), 
  \qquad s \geq 0,
\end{equation*}
and \cite{WangEtal95} found that, for $q \geq 1$, 
\begin{equation*}
  \mathcal{L}^{(q)} ( s ) = ( - 1 )^{q} \left( (1 - \nu ) s^{- \nu} \right)^{q} %\exp \left(- s^{1 - \nu} \right)
    \left[ \sum_{m=0}^{q-1} \Omega_{q, m} s^{- m ( 1 - \nu )} \right]
    \mathcal L (s),
\end{equation*}
where the $\Omega_{q, m}$'s are polynomials of degree $m$, given recursively by
\begin{align}
  \Omega_{q, 0} & = 1, 
    \nonumber \\
  \Omega_{q, m} & = \Omega_{q - 1, m} + \Omega_{q - 1, m - 1} \left\{ \frac{q - 1}{1 - \nu} - (q - m) \right\}, \hspace{0.5cm} m = 1, \ldots, q - 2, 
    \label{eq:PSOmegas}\\
  \Omega_{q, q - 1} & = (1 - \nu )^{1 - q} \frac{\Gamma ( q - (1 - \nu) )}{\Gamma ( \nu )}\cdot 
    \nonumber 
\end{align}

It follows that
\begin{equation}
\log \left( ( - 1 )^{q} \mathcal{L}^{( q )} ( s ) \right) = q \left( \log ( 1 - \nu ) - \nu \log ( s ) \right) + 
	\log \left[ \sum_{m=0}^{q-1} \Omega_{q, m} s^{- m ( 1 - \nu )} \right] - s^{1 - \nu}.
\label{eqn:possta}
\end{equation}

With clustered data, the Kendall's tau for positive stable distributed frailties is 
\begin{equation*}
  \tau = \nu \in (0, 1).
\end{equation*}
